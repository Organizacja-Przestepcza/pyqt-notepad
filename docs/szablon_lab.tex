% szablon sprawozdania z dnia 2024-10-01
\documentclass{article} % mwart , article
\usepackage[utf8]{inputenc}	
\usepackage{polski}	
\usepackage[T1]{fontenc}
\frenchspacing	
\usepackage{indentfirst}
% PAKIETY DO MODYFIKACJI SRONY
\usepackage{fancybox} 
\usepackage{geometry}
\geometry{
	total={170mm,250mm},
	left=20mm,
	top=20mm,
}
\usepackage{listings}
\usepackage{float}
% TABELA
\usepackage{multirow}
\usepackage{graphicx}
\usepackage{array}
\usepackage{makecell}
% INNE
\usepackage{color, colortbl}
\definecolor{Gray}{gray}{0.9}
\usepackage{lipsum}
% USTAWIENIA SUBSECTION
\usepackage[compact]{titlesec}
\titleformat{\subsection}[runin]
{\normalfont\large\bfseries}{\thesubsection}{1em}{}[{\\[0,5em]}]
\titlespacing*{\subsection}{1cm}{1em}{0em}
% dodanie FORCEINDENT
\newcommand{\forceindent}{\leavevmode{\parindent=1cm\indent}} %


% TYTUŁ, DATY ORAZ DANE OSOBY OPRACOWUJĄCEJ SPRAWOZDANIE
\def\AuthorFirstName{\uppercase{Magdalena}}
\def\AuthorLastName{\uppercase{Machacka}}
\def\Title{ \\ Praca z git i github\\ \\ }
\def\DateOfExecution{2025-11-24}
\def\DateOfDeliver{2025-11-24}
%
%
\begin{document}	
	%\fancypage{\setlength{\fboxsep}{0pt}\doublebox}{}	
	\noindent
	\def\arraystretch{1.5} \small
% TABELA Z DANYMI
	\begin{table}
		\resizebox{\textwidth}{!}{\begin{tabular}{|p{2cm}|p{4cm}|p{3cm}|p{3cm}|l|}
		\hline 
		\multirow{4}{*}{\makecell[l]{\includegraphics[width=2cm]{image/ModLogoPO}\\\includegraphics[width=2cm]{image/ModLogoWE}}} & PRZEDMIOT: & \multicolumn{3}{l|}{ \makecell[l]{\noalign{\vskip3pt}PRZEDMIOT WYBIERALNY XIV:\\ NARZ\k{E}DZIA INFORMATYCZNE W PRAKTYCE \\IN\.{Z}YNIERSKIEJ}}\tabularnewline
		\cline{2-5} \cline{3-5} \cline{4-5} \cline{5-5} 
		& KIERUNEK STUDIÓW: & INFORMATYKA & ROK STUDIÓW: & IV\tabularnewline
		\cline{2-5} \cline{3-5} \cline{4-5} \cline{5-5} 
		& ROK AKADEMISKI: & 2020/2021 & SEMESTR: & 7\tabularnewline
		\cline{2-5} \cline{3-5} \cline{4-5} \cline{5-5} 
		& TEMAT: & \multicolumn{3}{l|}{\makecell[l]{\noalign{\vskip3pt} \Title }}\tabularnewline
		\hline 
		IMI\k{E}: & \textbf{\AuthorFirstName} & \multicolumn{2}{l|}{DATA WYKONANIA \'{C}WICZENIA: }  & \DateOfExecution \tabularnewline
		\hline 
		NAZWISKO: & \textbf{\AuthorLastName} & \multicolumn{2}{l|}{DATA ODDANIA SPRAWOZDANIA:} & \DateOfDeliver \tabularnewline
		\hline 
		OCENA: & DATA: & \multicolumn{3}{l|}{UWAGI}\tabularnewline
		\hline \rowcolor{Gray}
		{\makecell[l]{ \\ \\ \\ \\}}&  & \multicolumn{3}{l|}{}\tabularnewline
		\hline 
		\end{tabular}}
	\end{table}
% POCZĄTEK SPRAWOZDANIA

\section{Informacje ogólne o projekcie}	
Nazwa projektu: \textbf{pyqt-notepad}
Skład zespołu:
\begin{itemize}
\item[-] Magdalena Machacka – lider projektu,
\item[-] Piotr Kurdziel – lider techniczny,
\item[-] Błażej Krasucki – dyrektor sprzedaży.
\end{itemize}
\section{Cel projektu}
	Celem projektu jest stworzenie aplikacji notatnika przy wykorzystaniu języka Python i biblioteki PyQt oraz pokazanie umiejętności współpracy nad projektem przy wykorzystaniu narzędzi Git oraz Github. W ramach projektu należy pokazać umiejętność tworzenia, utrzymywania oraz naprawiania zmian w kodzie. 
\section{Przebieg pracy}
	\subsection{Utworzenie repozytorium}
	Repozytorium utworzono za pomocą platformy \texttt{github.com}. Uczestnicy projektu sklonowali repozytorium, w celu współpracy, co zostało przedstawione na rysunku \ref{fig:gitclone}.
	
	\begin{figure}[H]
	\centering
	\includegraphics[width=0.7\linewidth]{image/git_clone}
	\caption{Sklonowanie repozytorium utworzonego za pośrednictwem platformy Github}
	\label{fig:gitclone}
	\end{figure}
	
	
	\subsection{Praca na branchach}
	\subsubsection{Branch sprawozdanie}
		\begin{lstlisting}[caption={Przejście na branch sprawozdanie}]
remote: Total 5 (delta 1), reused 5 (delta 1), pack-reused 0 (from 0)
Unpacking objects: 100% (5/5), 31.12 KiB | 3.11 MiB/s, done.
From https://github.com/Organizacja-Przestepcza/pyqt-notepad
* [new branch]      sprawozdanie -> origin/sprawozdanie
Already up to date.
(base) magda@mThinkPad-t480:~/Documents/Github/pyqt-notepad$ git branch 
* main
(base) magda@mThinkPad-t480:~/Documents/Github/pyqt-notepad$ git switch sprawozdanie 
branch 'sprawozdanie' set up to track 'origin/sprawozdanie'.
Switched to a new branch 'sprawozdanie'
		\end{lstlisting}
		
		\begin{lstlisting}[caption={Dodanie plików sprawozdania i commit}]
git add .
(base) magda@mThinkPad-t480:\~{}/Documents/Github/pyqt-notepad\$ git status
On branch sprawozdanie
Your branch is up to date with 'origin/sprawozdanie'.

Changes to be committed:
(use ``git restore --staged <file>...'' to unstage)
	new file:   docs/.gitignore
	new file:   docs/image/ModLogoPO.jpg
	new file:   docs/image/ModLogoWE.jpg
	new file:   docs/image/logotyp-politechnika-opolska-01.jpg
	renamed:    resources/images/git\_clone.png -> docs/images/git\_clone.png
	new file:   docs/szablon\_lab.tex

(base) magda@mThinkPad-t480:\~{}/Documents/Github/pyqt-notepad\$ git commit -m ``feat: add docs''
		\end{lstlisting}
		
	\subsection{Pull requesty i code review}
	\subsection{Konflikty i sposób rozwiązania}
	\subsection{Reakcja na zmianę wprowadzoną przez prowadzącego}
	\subsection{Utworzenie wersji (tagów)}
	
\section{Opis programu w Pythonie}
	\subsection{Funkcjonalności}
	\subsection{Sposób uruchomienia}
	\subsection{Wymagania}

\section{Podsumowanie współpracy}
	
	
\end{document}	
