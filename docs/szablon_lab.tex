% szablon sprawozdania z dnia 2024-10-01
\documentclass{article} % mwart , article
\usepackage[utf8]{inputenc}	
\usepackage{polski}	
\usepackage[T1]{fontenc}
\frenchspacing	
\usepackage{indentfirst}
% PAKIETY DO MODYFIKACJI SRONY
\usepackage{fancybox} 
\usepackage{geometry}
\geometry{
	total={170mm,250mm},
	left=20mm,
	top=20mm,
}
% TABELA
\usepackage{multirow}
\usepackage{graphicx}
\usepackage{array}
\usepackage{makecell}
% INNE
\usepackage{color, colortbl}
\definecolor{Gray}{gray}{0.9}
\usepackage{lipsum}
% USTAWIENIA SUBSECTION
\usepackage[compact]{titlesec}
\titleformat{\subsection}[runin]
{\normalfont\large\bfseries}{\thesubsection}{1em}{}[{\\[0,5em]}]
\titlespacing*{\subsection}{1cm}{1em}{0em}
% dodanie FORCEINDENT
\newcommand{\forceindent}{\leavevmode{\parindent=1cm\indent}} %


% TYTUŁ, DATY ORAZ DANE OSOBY OPRACOWUJĄCEJ SPRAWOZDANIE
\def\AuthorFirstName{MARCIN}
\def\AuthorLastName{KOWOL}
\def\Title{Zapoznanie z zasadami bezpieczeństwa obowiązującymi \\w laboratorium, potwierdzenie odbycia instruktażu. \\ Wprowadzenie do LV, instalacja urządzeń,\\ Nawigacja w LabVIEW.}
\def\DateOfExecution{2024-10-01}
\def\DateOfDeliver{2024-10-08}
%
%
\begin{document}	
	%\fancypage{\setlength{\fboxsep}{0pt}\doublebox}{}	
	\noindent
	\def\arraystretch{1.5} \small
% TABELA Z DANYMI
	\begin{table}
		\resizebox{\textwidth}{!}{\begin{tabular}{|p{2cm}|p{4cm}|p{3cm}|p{3cm}|l|}
		\hline 
		\multirow{4}{*}{\makecell[l]{\includegraphics[width=2cm]{image/ModLogoPO}\\\includegraphics[width=2cm]{image/ModLogoWE}}} & PRZEDMIOT: & \multicolumn{3}{l|}{ \makecell[l]{\noalign{\vskip3pt}PRZEDMIOT WYBIERALNY XIV:\\ NARZ\k{E}DZIA INFORMATYCZNE W PRAKTYCE \\IN\.{Z}YNIERSKIEJ}}\tabularnewline
		\cline{2-5} \cline{3-5} \cline{4-5} \cline{5-5} 
		& KIERUNEK STUDIÓW: & INFORMATYKA & ROK STUDIÓW: & IV\tabularnewline
		\cline{2-5} \cline{3-5} \cline{4-5} \cline{5-5} 
		& ROK AKADEMISKI: & 2020/2021 & SEMESTR: & 7\tabularnewline
		\cline{2-5} \cline{3-5} \cline{4-5} \cline{5-5} 
		& TEMAT: & \multicolumn{3}{l|}{\makecell[l]{\noalign{\vskip3pt} \Title }}\tabularnewline
		\hline 
		IMI\k{E}: & \textbf{\AuthorFirstName} & \multicolumn{2}{l|}{DATA WYKONANIA \'{C}WICZENIA: }  & \DateOfExecution \tabularnewline
		\hline 
		NAZWISKO: & \textbf{\AuthorLastName} & \multicolumn{2}{l|}{DATA ODDANIA SPRAWOZDANIA:} & \DateOfDeliver \tabularnewline
		\hline 
		OCENA: & DATA: & \multicolumn{3}{l|}{UWAGI}\tabularnewline
		\hline \rowcolor{Gray}
		{\makecell[l]{ \\ \\ \\ \\}}&  & \multicolumn{3}{l|}{}\tabularnewline
		\hline 
		\end{tabular}}
	\end{table}
% POCZĄTEK SPRAWOZDANIA

\section{Informacje ogólne o projekcie}	
Nazwa projektu: \textbf{pyqt-notepad}
Skład zespołu:
\begin{itemize}
\item Magdalena Machacka – lider projektu,
\item Piotr Kurdziel – lider techniczny,
\item Błażej Krasucki – dyrektor sprzedaży.
\end{itemize}
\section{Cel projektu}
	
\section{Przebieg pracy}
	\subsection{Utworzenie repozytorium}
	\subsection{Praca na branchach}
	\subsection{Pull requesty i code review}
	\subsection{Konflikty i sposób rozwiązania}
	\subsection{Reakcja na zmianę wprowadzoną przez prowadzącego}
	\subsection{Utworzenie wersji (tagów)}
	
\section{Opis programu w Pythonie}
	\subsection{Funkcjonalności}
	\subsection{Sposób uruchomienia}
	\subsection{Wymagania}

\section{Podsumowanie współpracy}
	
	
\end{document}	
